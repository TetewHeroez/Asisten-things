\documentclass[10pt,openany,a4paper]{article}
\usepackage{graphicx} 
\usepackage{multirow}
\usepackage{enumitem}
\usepackage{amssymb}
\usepackage{amsmath}
\usepackage{amsthm}
\usepackage{xcolor}
\usepackage{multicol}
\usepackage{multirow}
\usepackage{array}
\usepackage{animate}
\usepackage{amsthm}
\usepackage{caption}
\usepackage{minted}
\usepackage{fancyhdr}
\usepackage{geometry}
	\geometry{
		total = {160mm, 237mm},
		left = 30mm,
		right = 35mm,
		top = 35mm,
    bottom = 30mm,
    headheight=2cm
	}
\renewcommand{\headrulewidth}{0pt}

\graphicspath{{C:/Users/teoso/OneDrive/Documents/Tugas Kuliah/Template Math Depart/}}

\newcommand{\R}{\mathbb{R}}
\newcommand{\N}{\mathbb{N}}
\newcommand{\Z}{\mathbb{Z}}
\newcommand{\Q}{\mathbb{Q}}
\newcommand{\jawab}{\textbf{Solusi}:}

\definecolor{bg}{rgb}{0.95, 0.95, 0.92}

\setminted[java]{bgcolor=bg,fontsize=\small,frame=single,bgcolorpadding=1mm,escapeinside=||}

\title{\textbf{Week 1 Assigment}}
\date{15 Maret 2025}
\author{Teosofi H.A \& Hafidz M.}

\begin{document}
  \maketitle
  \pagenumbering{gobble}
  \section*{Tugas Mandiri}
  \begin{enumerate}
    \item Perhatikan kedua program berikut ini:
    \begin{multicols}{2}
      \begin{minted}{java}
public static int gcd(int a, int b) {
  while (a!=b) {
    if (a<b) b-=a;
    if (a>b) a-=b;
  }
  return a;
}
      \end{minted}
      \columnbreak
      \begin{minted}{java}
public static int lcm(int c, int d) {
  int a=c, b=d;
  while (c!=d) {
    if (c<d) c+=a;
    if (c>d) d+=b;
  }
  return c;
}
      \end{minted}
    \end{multicols}
    Buktikan atau sangkal bahwa masing-masing program menghasilkan output FPB dan KPK dari dua buah bilangan asli.

    \item Diberikan potongan program sebagai berikut:
    \begin{minted}{java}
long[][] Fufu = {{1,0},{0,1}};
long[][] Fafa = {{1,1},{1,0}};

for (int i=1; i<=n ; i++){
  Fufu = |\textcolor{blue}{multiply}|(Fufu,Fafa); //*
}
    \end{minted}
    \textcolor{blue}{*Note: \texttt{multiply} merupakan \textit{method} yang mengalikan dua buah matriks $2\times 2$.}

    Buktikan dengan induksi bahwa untuk setiap bilangan asli $n$, maka variabel \texttt{Fufu} akan menyimpan matriks
    \[\begin{bmatrix}
      F_{n+1} & F_n\\
      F_n & F_{n-1}
    \end{bmatrix}\]
    Dengan $F_n$ adalah bilangan Fibonacci ke-$n$.
  \end{enumerate}
\end{document}