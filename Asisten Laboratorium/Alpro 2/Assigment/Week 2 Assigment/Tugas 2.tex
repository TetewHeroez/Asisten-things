\documentclass[10pt,openany,a4paper]{article}
\usepackage{graphicx} 
\usepackage{multirow}
\usepackage{enumitem}
\usepackage{amssymb}
\usepackage{amsmath}
\usepackage{amsthm}
\usepackage{xcolor}
\usepackage{multicol}
\usepackage{multirow}
\usepackage{array}
\usepackage{animate}
\usepackage{amsthm}
\usepackage{caption}
\usepackage{minted}
\usepackage{fancyhdr}
\usepackage{geometry}
	\geometry{
		total = {160mm, 237mm},
		left = 30mm,
		right = 35mm,
		top = 35mm,
    bottom = 30mm,
    headheight=2cm
	}
\renewcommand{\headrulewidth}{0pt}

\graphicspath{{C:/Users/teoso/OneDrive/Documents/Tugas Kuliah/Template Math Depart/}}

\newcommand{\R}{\mathbb{R}}
\newcommand{\N}{\mathbb{N}}
\newcommand{\Z}{\mathbb{Z}}
\newcommand{\Q}{\mathbb{Q}}
\newcommand{\jawab}{\textbf{Solusi}:}

\definecolor{bg}{rgb}{0.95, 0.95, 0.92}

\setminted[java]{bgcolor=bg,fontsize=\small,frame=single,bgcolorpadding=1mm,escapeinside=||}

\title{\textbf{Week 2 Assigment}}
\date{22 Maret 2025}
\author{Teosofi H.A \& Hafidz M.}

\begin{document}
  \maketitle
  \pagenumbering{gobble}
  \section*{Tugas Mandiri}
  \begin{enumerate}
    \item Perhatikan program berikut:
    \begin{minted}{java}
public static int solution(int n) {
  int solutions = 0;
  for (int a = 0; a <= n; a++) 
      for (int b = 0; b <= n; b++) 
          for (int c = 0; c <= n; c++) 
              for (int d = 0; d <= n; c++) 
                  if (a + b + c + d == n) solutions++;
  return solutions;
}
    \end{minted}
    \begin{enumerate}
      \item Tentukan kompleksitas waktu dan big-$\mathcal{O}$ dari program di atas!
      \item\textbf{(Opsional)} Modifikasi program di atas agar memiliki kompleksitas waktu hingga mendekati $\mathcal{O}(1)$!  
    \end{enumerate}
    \item .
  \end{enumerate}
\end{document}