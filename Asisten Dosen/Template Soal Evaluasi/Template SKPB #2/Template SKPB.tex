\documentclass[11pt,openany,a4paper]{article}
\usepackage{amsmath, amsfonts, amssymb, amsthm}
\usepackage{tikz, pgfplots, tkz-euclide,calc}
    \usetikzlibrary{patterns,snakes,shapes.arrows,arrows.meta}
\usepackage{fancyhdr}
\usepackage{enumerate,enumitem}
\usepackage{cancel}
\usepackage{varwidth}
\usepackage{array}
\usepackage{animate}
\usepackage{multirow,multicol}
\usepackage{hyperref}
\hypersetup{
    colorlinks=true,
    linkcolor=blue,
    filecolor=magenta,      
    urlcolor=cyan,
    pdftitle={Overleaf Example},
    pdfpagemode=FullScreen,
    }
\usepackage{graphicx}
\graphicspath{{D:/Hada Touya/Asisten-things/Asisten Dosen/PPT Kalkulus/Gambar & Logo/}{C:/Users/teoso/OneDrive/Documents/Asisten Dosen & Lab/Asisten Laboratorium/Alpro 1/PPT/Graphicx/}{C:/Users/teoso/OneDrive/Documents/Tugas Kuliah/Template Math Depart/}}

% TAMBAHKAN PACKAGE SENDIRI KALAU KURANG

\usepackage{geometry}
\geometry{
	left = 20mm,
	right = 20mm,
	top = 30mm,
	bottom = 30mm,
}


\pagestyle{fancy}
\fancyhead{}
\fancyfoot{}
\fancyhead[r]{}
\fancyhead[l]{\fbox{\large{\textbf{SKPB - ITS}}}}
\renewcommand{\headrulewidth}{0pt}
\renewcommand{\footrulewidth}{0pt}

\newcommand{\R}{\mathbb{R}}
\newcommand{\N}{\mathbb{N}}
\newcommand{\C}{\mathbb{C}}
\newcommand{\Z}{\mathbb{Z}}
\newcommand{\Q}{\mathbb{Q}}
\newcommand{\cis}{\text{cis}}

\begin{document}

\begin{center}
    {\underline{\textbf{\MakeUppercase{Evaluasi Tengah Semester Bersama Genap 2024/2025}}}}
\end{center}

\begin{center}
    \begin{tabular}{lcl}
        Mata kuliah/SKS & : & Kalkulus 1 ( SM234101 ) / 3 SKS \\
        Hari, Tanggal   & : & Kamis, 17 Oktober 2024          \\
        Waktu           & : & 07.00-08.40 WIB (100 menit)     \\
        Sifat           & : & Tertutup                        \\
        Kelas           & : & 5-12, 101
    \end{tabular}
\end{center}

\noindent\rule{\textwidth}{2.pt}

\setlength{\parindent}{5pt}
\setlength{\parindent}{5pt}
\centering{Tuliskan: Nama, NRP, dan Nomor Kelas pada lembar jawaban Anda.}
\setlength{\parindent}{5pt}
\par \textbf{\small\MakeUppercase{dilarang membawa/menggunakan kalkulator dan alat komunikasi}}
\centering{\textbf{\MakeUppercase{dilarang memberikan/menerima jawaban selama ujian}}}
\par \centering{\textbf{"Setiap tindak kecurangan akan mendapat sanksi akademik."}}
\noindent\rule{\textwidth}{2.pt}

\begin{table}[h]
    \centering
    ETS Mengukur Kemampuan
    \begin{tabular}{|c|m{11cm}|c|c|}
        \hline
        CPL & CPMK                                                                                      & SOAL & BOBOT (\%) \\ \hline
        \multirow{5}{*}{2}
            & CPMK-1 Mampu menyelesaikan persamaan dan pertidaksamaan serta menssketsa grafik persamaan & 1    & 20         \\ \cline{2-4}
            & \multirow{4}{*}{CPMK-2 Mampu menentukan kekontinuan fungsi dan turunannya}                & 2    & 20         \\\cline{3-4}
            &                                                                                           & 3    & 20         \\ \cline{3-4}
            &                                                                                           & 4    & 20         \\ \cline{3-4}
            &                                                                                           & 5    & 20         \\ \hline
    \end{tabular}
\end{table}
{\centering\textbf{SOAL}}
% SOAL DI SINI YAA
\begin{enumerate}
    \item
\end{enumerate}

\fancyfoot{\begin{center}
        \rule{0.28\textwidth}{2.pt}$\quad$\textbf{Selamat Mengerjakan}$\quad$\rule{0.28\textwidth}{2.pt}
        \begin{quote}
            \centering
            \textit{``Jujur adalah kunci kesuksesan''}
        \end{quote}
    \end{center}}
\newpage

\begin{center}
    {\underline{\textbf{\MakeUppercase{Evaluasi Tengah Semester Bersama Genap 2024/2025}}}}
\end{center}

\begin{center}
    \begin{tabular}{lcl}
        Mata kuliah/SKS & : & Kalkulus 1 ( SM234101 ) / 3 SKS \\
        Hari, Tanggal   & : & Kamis, 12 Desember 2024         \\
        Waktu           & : & 13.30-15.10 WIB (100 menit)     \\
        Sifat           & : & Tertutup                        \\
        Kelas           & : & 34-46, 107, 108
    \end{tabular}
\end{center}

\noindent\rule{\textwidth}{2.pt}

\setlength{\parindent}{5pt}
\setlength{\parindent}{5pt}
\centering{Tuliskan: Nama, NRP, dan Nomor Kelas pada lembar jawaban Anda.}
\setlength{\parindent}{5pt}
\par \textbf{\small\MakeUppercase{dilarang membawa/menggunakan kalkulator dan alat komunikasi}}
\centering{\textbf{\MakeUppercase{dilarang memberikan/menerima jawaban selama ujian}}}
\par \centering{\textbf{"Setiap tindak kecurangan akan mendapat sanksi akademik."}}
\noindent\rule{\textwidth}{2.pt}

\begin{table}[h]
    \centering
    EAS Mengukur Kemampuan
    \begin{tabular}{|c|m{10.5cm}|c|c|}
        \hline
        CPL & CPMK                                                                                                          & SOAL & BOBOT (\%) \\ \hline
        \multirow{5}{*}{2}
            & CPMK-2 Mampu menentukan kekontinuan fungsi dan                                                                & 1    & 20         \\ \cline{3-4}
            & turunannya                                                                                                    & 2    & 20         \\\cline{2-4}
            & CPMK-3 Mampu menghitung integral melalui teorema fundamental kalkulus                                         & 3    & 20         \\ \cline{2-4}
            & CPMK-4 Mampu mengaplikasikan bentuk peubah kompleks dalam bentuk polar serta mencari akar-akar persamaannya   & 4    & 20         \\ \cline{2-4}
            & CPMK-5 Mampu menerapkan konsep matriks untuk menyelesaikan sistem persamaan linier dan menentukan nilai eigen & 5    & 20         \\ \hline
    \end{tabular}
\end{table}
\centering{\textbf{SOAL}}
% SOAL DI SINI YAA
\begin{enumerate}
    \item
\end{enumerate}

\end{document}