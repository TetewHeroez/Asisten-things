\documentclass{article}
\usepackage{graphicx}
\usepackage{amsmath, amsfonts, amssymb, amsthm}
\usepackage{tikz, pgfplots, tkz-euclide,calc}
    \usetikzlibrary{patterns,snakes,shapes.arrows}
\usepackage{fancyhdr}
\usepackage{enumerate,enumitem}
\usepackage{cancel}
\usepackage{varwidth}
\usepackage{fontawesome}
\usepackage{derivative}
\usepackage{xfrac}

% TAMBAHKAN PACKAGE SENDIRI KALAU KURANG

\usepackage{geometry}
\geometry{
	total = {160mm, 237mm},
	left = 25mm,
	right = 20mm,
	top = 30mm,
	bottom = 30mm,
}

\pagestyle{fancy}
\fancyhead{}
\fancyfoot{}
\fancyhead[c]{\large{\textbf{\color{blue}Script Pembahasan Soal Matematika - Deret Tak Hingga}}}
\fancyfoot[r]{\textit{\color{blue}By: Teosofi Hidayah Agung (5002221132)}}
\renewcommand{\headrulewidth}{0pt}
\renewcommand{\footrulewidth}{0pt}
\newcommand{\jawab}{\textbf{Jawab}:}

\begin{document}
\pagenumbering{gobble}
    \begin{enumerate}
        \item[2.] Tentukan apakah deret-deret berikut konvergen atau divergen. Jika konvergen, dapatkan nilainya.
        \begin{enumerate}
            \item[(b)] $\displaystyle \sum_{k=1}^{\infty} \dfrac{1}{\ln 2^k}$\\\\
            \jawab\\
            Kita tahu bahwa $\ln 2^k = k\ln 2$. Maka, deret tersebut dapat ditulis ulang sebagai
            \[\sum_{k=1}^{\infty} \dfrac{1}{k\ln 2} = \dfrac{1}{\ln 2}\sum_{k=1}^{\infty} \dfrac{1}{k}\]
            Deret $\displaystyle \sum_{k=1}^{\infty} \dfrac{1}{k}$ adalah deret harmonik merupakan deret yang divergen. Maka, deret diatas divergen.\\

            \item[(h)] $\displaystyle \sum_{k=1}^{\infty} \dfrac{4^{k+2}}{7^{k-1}}$\\\\
            \jawab\\
            Kita tahu bahwa $4^{k+2} = 4^2\cdot 4^k = 16\cdot 4^k$ dan $7^{k-1} = \dfrac{1}{7}\cdot 7^k$. Maka, deret tersebut dapat ditulis ulang sebagai
            \[\sum_{k=1}^{\infty} \dfrac{16\cdot 4^k}{7\cdot 7^k} = \dfrac{16}{7}\sum_{k=1}^{\infty} \left(\dfrac{4}{7}\right)^k\]
            Deret $\displaystyle \sum_{k=1}^{\infty} \left(\dfrac{4}{7}\right)^k$ adalah deret geometri dengan $a = 4/7$ dan $r = 4/7$. Karena $|r| < 1$, maka deret tersebut konvergen. Maka, deret diatas konvergen dan nilainya adalah
            \[\dfrac{16}{7}\sum_{k=1}^{\infty} \left(\dfrac{4}{7}\right)^k=\dfrac{16}{7}\cdot \dfrac{1}{1-\dfrac{4}{7}} = \dfrac{16}{7}\cdot \dfrac{7}{3} = \dfrac{16}{3}\]

            \item[(m)] $\displaystyle \sum_{k=1}^{\infty} (e/\pi)^{k-1}$\\\\
            \jawab\\
            Kita tahu bahwa $(e/\pi)^{k-1} = \dfrac{e^{k-1}}{\pi^{k-1}} = \dfrac{\pi}{e}\cdot \dfrac{e^k}{\pi^k}$. Maka, deret tersebut dapat ditulis ulang sebagai
            \[\sum_{k=1}^{\infty} \dfrac{\pi}{e}\cdot \left(\dfrac{e}{\pi}\right)^{k} = \dfrac{\pi}{e}\sum_{k=1}^{\infty} \left(\dfrac{e}{\pi}\right)^k\]
            Deret $\displaystyle \sum_{k=1}^{\infty} \left(\dfrac{e}{\pi}\right)^k$ adalah deret geometri dengan $a = e/\pi$ dan $r = e/\pi$. Perhatikan bahwa $e=2,78...$ dan 
            $\pi=3,14...$, maka $e<\pi\implies \dfrac{e}{\pi}<1$. Karena $|r| < 1$, maka deret tersebut konvergen dan nilainya adalah
            \[\dfrac{\pi}{e}\sum_{k=1}^{\infty} \left(\dfrac{e}{\pi}\right)^k=\dfrac{\pi}{e}\cdot\frac{1}{1-\dfrac{e}{\pi}}=\dfrac{\pi}{e}\cdot\frac{\pi}{\pi-e}=\frac{\pi^2}{e\pi-e^2}\]
        \end{enumerate}
        \item[4.] Dapatkan bentuk tertutup untuk jumlahan parsial ke-$n$ dari deret.
        \[\ln\frac{1}{2}+\ln\frac{2}{3}+\ln\frac{3}{4}+...+\ln\frac{n}{n+1}+...\]
        Dan tentukan apakah deret tersebut konvergen.\\
        \jawab\\
        Kita tahu bahwa $\ln\dfrac{k}{k+1} = \ln k - \ln (k+1)$. Maka, deret tersebut dapat ditulis ulang sebagai
        \[\ln 1 - \ln 2 + \ln 2 - \ln 3 + \ln 3 - \ln 4 + ... + \ln n - \ln (n+1) + ...\]
        Dengan mengelompokkan suku-suku yang sama, kita dapatkan
        \[\ln 1 - \ln 2 + \ln 2 - \ln 3 + \ln 3 - \ln 4 + ... + \ln n - \ln (n+1) = \ln 1 - \ln (n+1)\]
        Sehingga, jumlahan parsial ke-$n$ dari deret tersebut adalah
        \[S_n = \ln 1 - \ln (n+1) = -\ln (n+1)\]
        Dari informasi diatas, kita dapatkan untuk $n\to\infty$ adalah
        \[\lim_{n\to\infty}S_n=\lim_{n\to\infty}-\ln (n+1)=-\infty\]
        Sehingga, deret tersebut divergen.

        \item[6.] Tunjukkan bahwa:
        \begin{enumerate}
            \item $\displaystyle \sum_{k=2}^{\infty} \ln(1-1/k^2) = -\ln 2$\\
            \jawab
            \begin{flalign*}
                \sum_{k=2}^{\infty} \ln(1-1/k^2)&=\sum_{k=2}^{\infty} \ln\left(\frac{k^2-1}{k^2}\right)&\\
                &=\sum_{k=2}^{\infty} \ln(k^2-1)-\ln k^2&\\
                &=\sum_{k=2}^{\infty} \ln((k-1)(k+1))-2\ln k&\\
                &=\sum_{k=2}^{\infty} \ln(k-1)+\ln(k+1)-2\ln k&\\
                &=\sum_{k=2}^{\infty} \ln(k-1)-\ln k+\sum_{k=2}^{\infty}\ln(k+1)-\ln k&\\
                &=(\ln 1-\cancel{\ln 2+\ln 2-\ln 3+...})+(\cancel{\ln 3}-\ln 2+\cancel{\ln 4-\ln 3...})&\\
                &=-\ln 2\,\blacksquare
            \end{flalign*}

            \item[(f)] $\displaystyle \sum_{k=1}^{\infty} (-1)^{k+1}\dfrac{2k+1}{k(k+1)} = 1$\\
            \jawab
            \begin{flalign*}
                \sum_{k=1}^{\infty} (-1)^{k+1}\dfrac{2k+1}{k(k+1)}&=\sum_{k=1}^{\infty} (-1)^{k+1}\left(\dfrac{1}{k}+\dfrac{1}{k+1}\right)&\\
                &=\sum_{k=1}^{\infty} (-1)^{k+1}\dfrac{1}{k}+\sum_{k=1}^{\infty} (-1)^{k+1}\dfrac{1}{k+1}&\\
                &=\left(1-\frac{1}{2}+\frac{1}{3}-\frac{1}{4}\pm...\right)+\left(\frac{1}{2}-\frac{1}{3}+\frac{1}{4}\mp\right)&\\
                &=1\,\blacksquare
            \end{flalign*}
        \end{enumerate}
        \item[11.] Dapatkan semua nilai $x$ yang menyebabkan deret konvergen, dan untuk nilai-nilai $x$ ini dapatkan jumlahannya.
        \begin{enumerate}
            \item[(b)] $\displaystyle e^{-x}+e^{-2x}+e^{-3x}+e^{-4x}+e^{-5x}+...$\\
            \jawab\\
            Deret tersebut dapat ditulis ulang menjadi
            \[e^{-x}+(e^{-x})^2+(e^{-x})^3+(e^{-x})^4+(e^{-x})^5+...\]
            yang dimana merupakan deret geometri dengan $r=e^{-x}$. Deret diatas akan konvergen 
            jika $|r|<1$, Sehingga
            \[|e^{-x}|<1\implies e^{-x}<1\implies -x<0\implies x>0\]
            Sehingga, deret tersebut konvergen untuk $x>0$. Jumlahannya adalah ($a=r=e^{-x}$)
            \[e^{-x}+e^{-2x}+e^{-3x}+e^{-4x}+e^{-5x}+...=\dfrac{e^{-x}}{1-e^{-x}}\]
            \item[(f)] $\displaystyle \sin x-\frac{1}{2}\sin^2 x+\frac{1}{4}\sin^3 x-\frac{1}{8}\sin^4 x+...$\\
            \jawab\\
            Deret tersebut dapat ditulis ulang menjadi
            \[2\left(\frac{1}{2}\sin x-\frac{1}{4}(\sin x)^2+\frac{1}{8}(\sin x)^3-\frac{1}{16}(\sin x)^4+...\right)\]
            yang dimana merupakan deret geometri dengan $r=\dfrac{1}{2}\sin x$. Deret diatas akan konvergen 
            jika $|r|<1$, Sehingga
            \[\left|\dfrac{1}{2}\sin x\right|<1\implies \left|\sin x\right|<2\implies -2<\sin x<2\]
            Namun mengingat $\sin x$ memiliki nilai maksimum dan minimum yaitu $-1$ dan $1$, maka
            untuk $x\in\mathbb{R}$, deret tersebut konvergen. Jumlahannya adalah ($a=\sin x, r=\dfrac{1}{2}\sin x$)
            \[\sin x-\frac{1}{2}\sin^2 x+\frac{1}{4}\sin^3 x-\frac{1}{8}\sin^4 x+...=\dfrac{2\sin x}{2-\sin x}\]
        \end{enumerate}
    \end{enumerate}
\end{document}