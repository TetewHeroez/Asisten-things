\documentclass[10pt,openany,a4paper]{article}
\usepackage{amsmath, amsfonts, amssymb, amsthm}
\usepackage{tikz, pgfplots, tkz-euclide,calc}
    \usetikzlibrary{patterns,snakes,shapes.arrows}
\usepackage{fancyhdr}
\usepackage{enumerate,enumitem}
\usepackage{cancel}
\usepackage{varwidth}
\usepackage{array}
\usepackage{multirow,multicol}

% TAMBAHKAN PACKAGE SENDIRI KALAU KURANG

\usepackage{geometry}
\geometry{
	total = {160mm, 237mm},
	left = 20mm,
	right = 20mm,
	top = 30mm,
	bottom = 30mm,
}


\pagestyle{fancy}
\fancyhead{}
\fancyfoot{}
\fancyhead[r]{}
\fancyhead[l]{\fbox{\large{\textbf{SKPB - ITS}}}}
\renewcommand{\headrulewidth}{0pt}
\renewcommand{\footrulewidth}{0pt}


\begin{document}

    \begin{center}
	{\underline{\textbf{\MakeUppercase{Evaluasi Tengah Semester Bersama Genap 2024/2025}}}}
    \end{center}

    \begin{center}
	\begin{tabular}{lcl}
		Mata kuliah/SKS & : & Kalkulus 1 ( SM234101 ) / 3 SKS\\
		Hari, Tanggal & : & Kamis, 12 Desember 2024\\
		Waktu & : & 13.30-15.10 WIB (100 menit)\\
		Sifat & : & Tertutup\\
		Kelas & : & 34-46, 107, 108
	\end{tabular}
    \end{center}
	
    \noindent\rule{\textwidth}{2.pt}
	
    \setlength{\parindent}{5pt}
    \setlength{\parindent}{5pt}
    \centering{Tuliskan: Nama, NRP, dan Nomor Kelas pada lembar jawaban Anda.}
    \setlength{\parindent}{5pt}
    \par \textbf{\MakeUppercase{dilarang membawa/menggunakan kalkulator dan alat komunikasi}}
    \centering{\textbf{\MakeUppercase{dilarang memberikan/menerima jawaban selama ujian}}}
    \par \centering{\textbf{"Setiap tindak kecurangan akan mendapat sanksi akademik."}}
    \noindent\rule{\textwidth}{2.pt}
    
	\begin{table}[h]
        \centering
        EAS Mengukur Kemampuan
        \begin{tabular}{|c|m{10.5cm}|c|c|}
            \hline
            CPL & CPMK & SOAL & BOBOT (\%) \\ \hline
            \multirow{5}{*}{2} 
            & CPMK-2 Mampu menentukan kekontinuan fungsi dan & 1 & 20 \\ \cline{3-4}
            & turunannya & 2 & 20 \\\cline{2-4}
            & CPMK-3 Mampu menghitung integral melalui teorema fundamental kalkulus & 3 & 20 \\ \cline{2-4}
            & CPMK-4 Mampu mengaplikasikan bentuk peubah kompleks dalam bentuk polar serta mencari akar-akar persamaannya & 4 & 20 \\ \cline{2-4}
            & CPMK-5 Mampu menerapkan konsep matriks untuk menyelesaikan sistem persamaan linier dan menentukan nilai eigen & 5 & 20 \\ \hline
        \end{tabular}
    \end{table}
    \centering{\textbf{SOAL}}
% SOAL DI SINI YAA
    \begin{enumerate}
        \item 
    \end{enumerate}
    \begin{center}
        \rule{0.28\textwidth}{2.pt}$\quad$\textbf{Selamat Mengerjakan}$\quad$\rule{0.28\textwidth}{2.pt}
    \end{center}
\end{document}