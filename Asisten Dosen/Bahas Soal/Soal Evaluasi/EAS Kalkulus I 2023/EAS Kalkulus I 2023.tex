\documentclass[11pt,openany,a4paper]{article}
\usepackage{amsmath, amsfonts, amssymb, amsthm}
\usepackage{tikz, pgfplots, tkz-euclide,calc}
    \usetikzlibrary{patterns,snakes,shapes.arrows}
\usepackage{fancyhdr}
\usepackage{enumerate,enumitem}
\usepackage{cancel}
\usepackage{varwidth}
\usepackage{hyperref}
\hypersetup{
    colorlinks=true,
    linkcolor=blue,
    filecolor=magenta,      
    urlcolor=cyan,
    }

% TAMBAHKAN PACKAGE SENDIRI KALAU KURANG

\usepackage{geometry}
\geometry{
	total = {160mm, 230mm},
	left = 16mm,
	right = 16mm,
	top = 30mm,
	bottom = 30mm,
}


\pagestyle{fancy}
\fancyhead{}
\fancyfoot{}
\fancyhead[r]{}
\fancyhead[l]{\fbox{\large{\textbf{SKPB - ITS}}}}
\renewcommand{\headrulewidth}{0pt}
\renewcommand{\footrulewidth}{0pt}


\begin{document}

    \begin{center}
	{\underline{\textbf{\MakeUppercase{Evaluasi Akhir Semester Bersama Gasal 2023/2024}}}}
    \end{center}

    \begin{center}
	\begin{tabular}{lcl}
		Mata kuliah/SKS & : & Kalkulus 1 ( SM234101 ) / 3 SKS\\
		Hari, Tanggal & : & Senin, 11 Desember 2023\\
		Waktu & : & 07.00-08.40 WIB (100 menit)\\
		Sifat & : & Tertutup\\
		Kelas & : & 4-10
	\end{tabular}
    \end{center}
	
    \noindent\rule{\textwidth}{2.pt}
	
    \setlength{\parindent}{5pt}
    \par Diberikan 5 soal, dengan bobot nilai masing-masing soal sama dan boleh dikerjakan tidak berurutan.
    \setlength{\parindent}{5pt}
    \centering{Tuliskan: Nama, NRP, dan Nomor Kelas pada lembar jawaban Anda.}
    \setlength{\parindent}{5pt}
    {\small
    \par \textbf{\MakeUppercase{Dilarang membawa/menggunakan kalkulator dan alat komunikasi}}
    \centering{\textbf{\MakeUppercase{dilarang memberikan/menerima jawaban selama ujian}}}}
    \par \centering{\textbf{"Setiap tindak kecurangan akan mendapat sanksi akademik."}}
	
    \noindent\rule{\textwidth}{2.pt}
	
% SOAL DI SINI YAA
    \begin{enumerate}
	\item Kotak empat persegi panjang tertutup dengan dasar bujur sangkar mempunyai volume \(2560 \text{ cm}^3\).  
  Harga bahan untuk bagian atas dan bawah adalah Rp. \(4.000\) per \(\text{cm}^2\), dan untuk sisinya adalah Rp. \(6.000\) per \(\text{cm}^2\).  
  Tentukan biaya terkecil untuk membuat kotak tersebut.  

        \item Diberikan fungsi \( f(x) = 5 + 4x^3 - x^4 \).  
        \begin{enumerate}
            \item Tentukan selang di mana fungsi \( f(x) \) naik atau turun.
            \item Tentukan titik ekstrem relatif fungsi tersebut.
            \item Tentukan selang kecekungan fungsi \( f(x) \) dan titik belok (jika ada).
            \item Sketsa grafiknya.
        \end{enumerate}

        \item Hitung integral  
        \[
        \int_{0}^{4} \frac{4x}{\sqrt{2x+1}} \, dx
        \]
        
        \item Hitung dan nyatakan dalam bentuk \( z = x + iy \) dari  
        \[
        \frac{(2 \text{ cis } 20^\circ)^5}{(2 \text{ cis } 40^\circ)^4}
        \]

        \item Selesaikan sistem persamaan linear berikut:  
        \[
        \begin{aligned}
            2x - 3y + 3z &= -17 \\
            x + 2y - 2z &= 9 \\
            3x + y + 2z &= -7
        \end{aligned}
        \]
        dengan eliminasi Gauss-Jordan
    \end{enumerate}
	
  %   \vspace*{2cm}
  %   \hrulefill\textbf{ Selamat Mengerjakan }\hrulefill

  %   \begin{center}
	% "\textit{Jujur adalah kunci kesuksesan}"
  %   \end{center}
    \fancyfoot{
    \hrulefill\textbf{ Selamat Mengerjakan }\hrulefill

    \begin{center}
	    ``\textit{Jujur adalah kunci kesuksesan}''
    \end{center}}

    \newpage
    \fancyhead[L]{\textit{Solution By: \hyperlink{https://github.com/TetewHeroez}{Tetew}}}
    % \fancyfoot[R]{\animategraphics[autoplay,loop,width=0.1\textwidth]{15}{Kuru Kuru Herta/kuru kuru-}{0}{5}}
    \fancyfoot{}
    {\centering\textbf{SOLUSI}}
    \renewcommand{\arraystretch}{1.5}
    \renewcommand{\headrulewidth}{1pt}
    \begin{enumerate}
        \item Misalkan $h$ adalah tinggi dari kotak dan $s$ merupakan panjang sisi alas kotak, maka didapatkan hubungan
        \[
            V = s^2 h = 2560 \implies h = \frac{2560}{s^2}
        \]
        Selanjutnya harga kotak bergantung pada $s$ dan $h$ yang dimana untuk harga alas dan penutupnya adalah Rp. \(4.000s^2\), sedangkan harga sisi-sisinya adalah Rp. \(6.000sh\). Karena kotak memiliki 4 sisi dan alas atas dan bawah, maka harga total kotak dapat dirumuskan sebagai
        \begin{align*}
            C(s) &= 2\cdot 4.000s^2 + 4\cdot 6.000sh \\
            C(s) &= 8.000s^2 + 24.000s\left(\frac{2560}{s^2}\right) \\
            C(s) &= 8.000s^2 + \frac{61.440.000}{s} 
        \end{align*}
        Kemudian untuk meminimumkan biaya, kita perlu mencari titik kritis dari fungsi biaya $C(s)$ dengan cara mencari turunan pertama dari $C(s)$
        \begin{align*}
            C'(s) &= 16.000s - \frac{61.440.000}{s^2} 
        \end{align*}
        Lanjut dengan menentukan $s$ yang memenuhi $C'(s) = 0$ untuk mencari titik kritis
        \begin{align*}
            16.000s - \frac{61.440.000}{s^2} &= 0 \\
            16.000s^3 &= 61.440.000 \\
            s^3 &= \frac{61.440.000}{16.000} \\
            s^3 &= 3840 \\
            s &= \sqrt[3]{3840}=4\sqrt[3]{60}
        \end{align*}
        Untuk menentukan apakah $s$ adalah titik minimum, kita perlu mencari turunan kedua dari $C(s)$
        \begin{align*}
            C''(s) &= 16.000 + \frac{122.880.000}{s^3}
        \end{align*}
        Karena $C''(s) > 0$ untuk semua $s > 0$, maka $s=4\sqrt[3]{60}$ adalah absis dari titik minimum. Dengan demikian, biaya minimumnya dalam rupiah adalah
        \begin{align*}
            C(4\sqrt[3]{60}) &= 8.000(4\sqrt[3]{60})^2 + \frac{61.440.000}{4\sqrt[3]{60}} \\
            &= 8.000\cdot 16\cdot 60^{2/3} + \frac{61.440.000}{4\sqrt[3]{60}} \\
            &= 128.000\cdot 60^{2/3} + \frac{61.440.000}{4\sqrt[3]{60}} \\
            &= \boxed{768000\sqrt[3]{450}}
        \end{align*}
        \item .
        \item .
        \item .
        \item Langkah pertama adalah menuliskan sistem persamaan dalam bentuk \textit{augmented matrix} 
        \[
        \begin{bmatrix}
            2 & -3 & 3 & | & -17 \\
            1 & 2 & -2 & | & 9 \\
            3 & 1 & 2 & | & -7
        \end{bmatrix}
        \]
        Selanjutnya kita akan melakukan OBE sehingga matrix tersebut menjadi bentuk eselon tereduksi
        \begin{align*}
          &\begin{bmatrix}
              2 & -3 & 3 & | & -17 \\
              1 & 2 & -2 & | & 9 \\
              3 & 1 & 2 & | & -7
          \end{bmatrix}
          \begin{matrix}
              \xrightarrow{B_1 \Leftrightarrow B_2} 
          \end{matrix}
          \begin{bmatrix}
              1 & 2 & -2 & | & 9 \\
              2 & -3 & 3 & | & -17 \\
              3 & 1 & 2 & | & -7
          \end{bmatrix}
          \begin{matrix}
              {\scriptstyle B_2 - 2B_1}\\
              \xrightarrow{B_3-3B_1}
          \end{matrix}
          \begin{bmatrix}
              1 & 2 & -2 & | & 9 \\
              0 & -7 & 7 & | & -35 \\
              0 & -5 & 8 & | & -34
          \end{bmatrix}\\
          &\begin{matrix}
            \xrightarrow{-\frac{1}{7}B_2}
          \end{matrix}
          \begin{bmatrix}
              1 & 2 & -2 & | & 9 \\
              0 & 1 & -1 & | & 5 \\
              0 & -5 & 8 & | & -34
          \end{bmatrix}
          \begin{matrix}
              \xrightarrow{B_3 + 5B_2}
          \end{matrix}
          \begin{bmatrix}
              1 & 2 & -2 & | & 9 \\
              0 & 1 & -1 & | & 5 \\
              0 & 0 & 3 & | & -9
          \end{bmatrix}
          \begin{matrix}
              \xrightarrow{\frac{1}{3}B_3}
          \end{matrix}
          \begin{bmatrix}
              1 & 2 & -2 & | & 9 \\
              0 & 1 & -1 & | & 5 \\
              0 & 0 & 1 & | & -3
          \end{bmatrix}\\
          &\begin{matrix}
              {\scriptstyle B_2 + B_3}\\
              \xrightarrow{B_1 + 2B_3}
          \end{matrix}
          \begin{bmatrix}
              1 & 2 & 0 & | & 3 \\
              0 & 1 & 0 & | & 2 \\
              0 & 0 & 1 & | & -3
          \end{bmatrix}
          \begin{matrix}
              \xrightarrow{B_1-2B_2}
          \end{matrix}
          \begin{bmatrix}
              1 & 0 & 0 & | & -1 \\
              0 & 1 & 0 & | & 2 \\
              0 & 0 & 1 & | & -3
          \end{bmatrix}
        \end{align*}
        Dengan demikian, kita mendapatkan solusi dari sistem persamaan linear tersebut adalah $x = -1$, $y = 2$, dan $z = -3$.
    \end{enumerate}
\end{document}