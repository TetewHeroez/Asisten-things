\documentclass[11pt,openany,a4paper]{article}
\usepackage{amsmath, amsfonts, amssymb, amsthm}
\usepackage{tikz, pgfplots, tkz-euclide,calc}
    \usetikzlibrary{patterns,snakes,shapes.arrows}
    \tikzset{Arrow Style/.style={text=black, font=\boldmath}}
    \newcommand{\tikzmark}[1]{%
        \tikz[overlay, remember picture, baseline] \node (#1) {};%
    }

    \newcommand*{\XShift}{0.5em}
    \newcommand*{\YShift}{0.5ex}

    \NewDocumentCommand{\DrawArrow}{s O{} m m m}{%
        \begin{tikzpicture}[overlay,remember picture]
            \draw[->, thick, Arrow Style, #2] 
                    ($(#3.west)+(\XShift,\YShift)$) -- 
                    ($(#4.east)+(-\XShift,\YShift)$)
            node [midway,above] {#5};
        \end{tikzpicture}%
    }
\usepackage{fancyhdr}
\usepackage{enumerate,enumitem}
\usepackage{cancel}
\usepackage{polynom}
\makeatletter
\patchcmd{\pld@ArrangeResult}{\bigr)}{
    \,\tikz[overlay]{
        \coordinate (A) at (0,-3pt);
        \coordinate (B) at (0,\normalbaselineskip-0.8pt);
        \draw (A) to[in=-40, out=40, looseness=1] (B);
    }
}{}{}
\makeatother
\makeatletter
\patchcmd{\pld@ArrangeResult}{\tw@-\pld@maxcol}{
  \tw@-\pld@maxcol
}{}{}
\makeatother
\usepackage{varwidth}
\usepackage{hyperref}
\hypersetup{
    colorlinks=true,
    linkcolor=blue,
    filecolor=magenta,      
    urlcolor=cyan,
    }

% TAMBAHKAN PACKAGE SENDIRI KALAU KURANG

\usepackage{geometry}
\geometry{
	total = {160mm, 237mm},
	left = 18mm,
	right = 13mm,
	top = 30mm,
	bottom = 30mm,
}


\pagestyle{fancy}
\fancypagestyle{problems}{
  \fancyhead{}
  \fancyfoot{}
  \fancyhead[r]{}
  \fancyhead[l]{\fbox{\large{\textbf{SKPB - ITS}}}}
  \renewcommand{\headrulewidth}{0pt}
  \renewcommand{\footrulewidth}{0pt}
}

\renewcommand{\headrulewidth}{0pt}
\renewcommand{\footrulewidth}{0pt}
\renewcommand{\hrulefill}{\leavevmode\leaders\hrule height 2pt\hfill\kern0pt}


\fancypagestyle{solution}{
  \pagestyle{fancy}
  \newpage
  \fancyhead[L]{\textit{Solution By: \hyperlink{https://github.com/TetewHeroez}{Tetew Heroez}}}
  % \fancyfoot[R]{\animategraphics[autoplay,loop,width=0.1\textwidth]{15}{Kuru Kuru Herta/kuru kuru-}{0}{5}}
  \fancyfoot{}
  \renewcommand{\headrulewidth}{1pt}
}

\newcommand{\Hrule}{\stackrel{\text{\raisebox{1.5pt}{\textcircled{\raisebox{-.9pt} {L}}}}}{=}}

\begin{document}
\pagestyle{problems}
    \begin{center}
	{\underline{\textbf{\MakeUppercase{Evaluasi Tengah Semester Bersama Genap 2024/2024}}}}
    \end{center}

    \begin{center}
	\begin{tabular}{lcl}
		Mata kuliah/SKS & : & Kalkulus 2 ( SM234201 ) / 3 SKS\\
		Hari, Tanggal & : & Rabu, 24 April 2024\\
		Waktu & : & 07.00-08.40 WIB (100 menit)\\
		Sifat & : & Tertutup\\
		Kelas & : & 1-13, 101
	\end{tabular}
    \end{center}
	
    \noindent\rule{\textwidth}{2.pt}
	
    \setlength{\parindent}{5pt}
    \par Diberikan 5 soal, dengan bobot nilai masing-masing soal sama dan boleh dikerjakan tidak berurutan.
    \setlength{\parindent}{5pt}
    \centering{Tuliskan: Nama, NRP, dan Nomor Kelas pada lembar jawaban Anda.}
    \setlength{\parindent}{5pt}
    {\small
    \par \textbf{\MakeUppercase{Dilarang membawa/menggunakan kalkulator dan alat komunikasi}}
    \centering{\textbf{\MakeUppercase{dilarang memberikan/menerima jawaban selama ujian}}}}
    \par \centering{\textbf{"Setiap tindak kecurangan akan mendapat sanksi akademik."}}
	
    \noindent\rule{\textwidth}{2.pt}
	
% SOAL DI SINI YAA
\begin{enumerate}
  \item
    Jika 
    $
      \dfrac{d}{dx}\left(\sec^{-1}u\right)
      = \dfrac{1}{u\sqrt{u^2-1}}\dfrac{du}{dx}
    $
    untuk \(u\) fungsi \(x\), maka dapatkan \(\displaystyle \frac{dy}{dx}\) jika 
    $
      y = \sec^{-1}\left(e^{-3x}\right).
    $
    
  \item Hitung integral
    \[
      \int_{\frac14}^{\frac12} \frac{e^p}{x^2}\,dx
    \]
    dengan $p=\dfrac{1}{2x}$
    
  \item Hitung integral
    \[
      \int \sin^2(2x)\,\cos^3(2x)\,dx.
    \]
    
  \item Hitung integral
    \[
      \int \frac{x^3}{x^2 - 5x - 6}\,dx.
    \]
    
  \item Dapatkan
    \[
      \lim_{x\to 0}\left(\frac{1}{x} - \frac{1}{e^x - 1}\right).
    \]
\end{enumerate}
	
    \fancyfoot{
    \hrulefill\textbf{ Selamat Mengerjakan }\hrulefill

    \begin{center}
	    ``\textit{Jujur adalah kunci kesuksesan}''
    \end{center}}

    \newpage
    \pagestyle{solution}
    {\centering\textbf{SOLUSI}}
    
    \begin{enumerate}
        \item Dari informasi yang diberikan, dapat dimisalkan bahwa
        \[
          u = e^{-3x} \implies \frac{du}{dx} = -3e^{-3x}
        \]
        Sehingga, kita dapatkan
        \[
          \frac{dy}{dx} = \dfrac{d}{dx}\left(\sec^{-1}u\right) = \frac{1}{u\sqrt{u^2-1}}\frac{du}{dx} = \frac{1}{\cancel{e^{-3x}}\sqrt{e^{-6x}-1}} \cdot (-3\cancel{e^{-3x}}) = \boxed{-\frac{3}{\sqrt{e^{-6x}-1}}}.
        \]

        \item Soal tersebut telah dipermudahkan dengan memberikan \text{clue} yaitu menggunakan substitusi 
        \[
          p = \frac{1}{2x} \implies 2dp = -\frac{1}{x^2}dx
        \]
        dan untuk batas integralnya, kita dapatkan
        \begin{align*}
          x = \frac{1}{4} &\implies p = \frac{1}{2(\frac{1}{4})} = 2 \\
          x = \frac{1}{2} &\implies p = \frac{1}{2(\frac{1}{2})} = 1
        \end{align*}
        Sehingga integral tersebut dapat ditulis ulang menjadi
        \begin{align*}
          \int_{\frac14}^{\frac12} \frac{e^{\frac{1}{2x}}}{x^2}\,dx &= \int_{2}^{1} e^p (-2dp) = 2\int_{1}^{2} e^p \, dp = 2\left[e^p\right]_{1}^{2} = \boxed{2\left(e^2 - e\right) } 
        \end{align*}
        
        \item Ingat bahwa $\cos^2(\theta) = 1 - \sin^2(\theta)$, sehingga integral tersebut dapat ditulis ulang menjadi\footnote{Anda juga dapat menggunakan rumus yang ada pada buku diktat Kalkulus II}
        \begin{align*}
          \int \sin^2(2x)\,\cos^3(2x)\,dx&= \int \sin^2(2x)\,\cos^2(2x) \cos(2x)\,dx \\
          &= \int \sin^2(2x)\left(1 - \sin^2(2x)\right)\cos(2x)\,dx\\
          &= \int \left[\sin^2(2x)-\sin^4(2x)\right]\cos(2x)\,dx
        \end{align*}
        Dengan menggunakan substitusi $u = \sin(2x)\implies \dfrac{1}{2}du= \cos(2x)\,dx$, diperoleh
        \begin{align*}
          \int \left[\sin^2(2x)-\sin^4(2x)\right]\cos(2x)\,dx &= \frac{1}{2}\int \left[u^2 - u^4\right]\,du \\
          &= \frac{1}{2}\left[\frac{u^3}{3} - \frac{u^5}{5}\right] + C \\
          &= \boxed{\frac{1}{6}\sin^3(2x) - \frac{1}{10}\sin^5(2x) + C}
        \end{align*}
        \item Karena bentuk pecahan rasional tersebut memiliki pembilang yang derajatnya lebih besar dari penyebutnya, maka kita harus menyederhanakannya terlebih dahulu. 
        
        Disini akan digunakan metode pembagian porogapit untuk menyederhanakan bentuk rasional tersebut.
        \begin{center}
          \polylongdiv{x^3}{x^2-5x-6}
        \end{center}
        Fungsi rasional tersebut dapat ditulis ulang menjadi
        \[
          \frac{x^3}{x^2 - 5x - 6} = x + 5 + \frac{31x+30}{x^2 - 5x - 6}
        \]
        Sekarang kita tinjau pecahan rasional
        \[
          \frac{31x+30}{x^2 - 5x - 6} = \frac{31x+30}{(x-6)(x+1)} = \frac{A}{x-6} + \frac{B}{x+1} = \frac{A(x+1) + B(x-6)}{(x-6)(x+1)}
        \]
        Dengan menyamakan pembilang, kita dapatkan
        \[
          31x + 30 = A(x+1) + B(x-6)
        \]
        Subtitusikan $x=6$ dan $x=-1$ untuk mendapatkan nilai $A$ dan $B$\footnote{$x$ yang dipilih boleh saja sembarang bilangan, namun akan lebih mudah jika kita memilih $x$ yang membuat salah satu ekspresi menjadi nol.}
        \begin{align*}
          x = 6 &\implies 31(6) + 30 = A(6+1) + B(0) \\
          &\implies 216 = 7A \implies A = \frac{216}{7} \\
          x = -1 &\implies 31(-1) + 30 = A(0) + B(-1-6) \\
          &\implies -1 = -7B \implies B = \frac{1}{7}    
        \end{align*}
        Sehingga integral pada soal dapat dituliskan sebagai
        \begin{align*}
          \int \frac{x^3}{x^2 - 5x - 6}\,dx &= \int x + 5 + \frac{216}{7(x-6)} + \frac{1}{7(x+1)}\,dx \\
          &= \int (x + 5)\,dx + \frac{216}{7}\int \frac{1}{x-6}\,dx + \frac{1}{7}\int \frac{1}{x+1}\,dx \\
          &= \boxed{\frac{x^2}{2} + 5x + \frac{216}{7}\ln|x-6| + \frac{1}{7}\ln|x+1| + C} \\
        \end{align*}

        \item Ketika $x\to 0$ ekspresi diatas menjadi bentuk tak tentu $\infty - \infty$, sehingga kita perlu mengubah ekspresi fungsinya agar mendapatkan bentuk $\dfrac{0}{0}$ atau $\dfrac{\infty}{\infty}$.\footnote{L'Hôpital hanya bisa digunakan pada bentuk $\frac{0}{0}$ atau $\frac{\infty}{\infty}$}
        
        Perhatikan bahwa
        \begin{align*}
          \lim_{x\to 0}\left(\frac{1}{x} - \frac{1}{e^x - 1}\right) &= \lim_{x\to 0}\left(\frac{e^x - 1 - x}{x(e^x - 1)}\right) = \frac{0}{0}
        \end{align*}
        oleh karena itu kita dapat menggunakan L'Hôpital untuk menyelesaikan limit tersebut.
        \begin{align*}
          \lim_{x\to 0}\left(\frac{e^x - 1 - x}{x(e^x - 1)}\right)\Hrule\lim_{x\to 0}\left(\frac{e^x - 1}{e^x - 1 + xe^x}\right)\Hrule \lim_{x\to 0}\left(\frac{e^x}{2e^x + xe^x}\right) = \frac{1}{2\cdot1+0\cdot1}=\boxed{\frac{1}{2}}
        \end{align*}
    \end{enumerate}

    \newpage
    \pagestyle{problems}

    \begin{center}
	{\underline{\textbf{\MakeUppercase{Evaluasi Tengah Semester Bersama Genap 2024/2024}}}}
    \end{center}

    \begin{center}
	\begin{tabular}{lcl}
		Mata kuliah/SKS & : & Kalkulus 2 ( SM234201 ) / 3 SKS\\
		Hari, Tanggal & : & Rabu, 24 April 2024\\
		Waktu & : & 09.00-10.40 WIB (100 menit)\\
		Sifat & : & Tertutup\\
		Kelas & : & 15-27, 102
	\end{tabular}
    \end{center}
	
    \noindent\rule{\textwidth}{2.pt}
	
    \setlength{\parindent}{5pt}
    \par Diberikan 5 soal, dengan bobot nilai masing-masing soal sama dan boleh dikerjakan tidak berurutan.
    \setlength{\parindent}{5pt}
    \centering{Tuliskan: Nama, NRP, dan Nomor Kelas pada lembar jawaban Anda.}
    \setlength{\parindent}{5pt}
    {\small
    \par \textbf{\MakeUppercase{Dilarang membawa/menggunakan kalkulator dan alat komunikasi}}
    \centering{\textbf{\MakeUppercase{dilarang memberikan/menerima jawaban selama ujian}}}}
    \par \centering{\textbf{"Setiap tindak kecurangan akan mendapat sanksi akademik."}}
	
    \noindent\rule{\textwidth}{2.pt}
	
% SOAL DI SINI YAA
\begin{enumerate}
  \item Dapatkan turunan dari \( f(x) = e^x \tan^{-1} x \).
  
  \item Hitung integral
  \[
  \int \frac{e^x}{e^x + 1} \, dx.
  \]
  
  \item Hitung integral
  \[
  \int \ln(t^2 + 1) \, dt.
  \]
  
  \item Hitung integral
  \[
  \int \frac{2x^2 - 2x - 1}{x^2 + x^3} \, dx.
  \]
  
  \item Selesaikan integral tak wajar
  \[
  \int_0^{\pi/6} \frac{\cos x}{\sqrt{1 - 2 \sin x}} \, dx.
  \]
\end{enumerate}
	
    \fancyfoot{
    \hrulefill\textbf{ Selamat Mengerjakan }\hrulefill

    \begin{center}
	    ``\textit{Jujur adalah kunci kesuksesan}''
    \end{center}}

    \newpage
    \pagestyle{solution}
    {\centering\textbf{SOLUSI}}
    \begin{enumerate}
        \item Ingat bahwa $\dfrac{d}{dx}\tan^{-1} x = \dfrac{1}{1+x^2}$. Kemudian dengan memisalkan $u = e^x$ dan $v=\tan^{-1} x$, maka dengan aturan perkalian pada diferensiasi didapatkan
        \begin{align*}
          f'(x) &= u'v + uv' \\
          &= e^x \tan^{-1} x + e^x \cdot \frac{1}{1+x^2} \\
          &= \boxed{e^x \left( \tan^{-1} x + \frac{1}{1+x^2} \right)}
        \end{align*}
        \item Gunakan teknik substitusi, yaitu $u = e^x + 1$, sehingga $du = e^x dx$. Akhirnya diperoleh
        \begin{align*}
          \int \frac{e^x}{e^x + 1} \, dx &= \int \frac{du}{u} = \ln|u| + C = \ln|e^x + 1| + C = \boxed{\ln(e^x + 1) + C}
        \end{align*}
        \item Gunakan teknik integral parsial dengan memisalkan
        \begin{align*}
          u &= \ln(t^2 + 1) \implies du = \frac{2t}{t^2 + 1} dt \\
          dv &= dt \implies v = t
        \end{align*}
        Selanjutnya didapatkan ekspresi
        \begin{align*}
          \int \ln(t^2 + 1) \, dt &= uv - \int v \, du = t \ln(t^2 + 1) - \int \frac{2t^2}{t^2 + 1} dt = t \ln(t^2 + 1) - 2\int \left(1-\frac{1}{t^2 + 1} \right)dt \\
        \end{align*}
        Langkah terakhir adalah menyelesaikan integral yang tersisa, yaitu
        \begin{align*}
          \int \left(1-\frac{1}{t^2 + 1} \right)dt &= t - \tan^{-1} t + C 
        \end{align*}
        Jadi didapatkan kesimpulan\footnote{Jika ragu dengan hasil akhir yang sekarang, anda dapat melakukan diferensiasi pada hasil akhir dan cocokkan dengan fungsi awal.}
        \begin{align*}
          \int \ln(t^2 + 1) \, dt &= \boxed{t \ln(t^2 + 1) - 2t - 2\tan^{-1} t + C}
        \end{align*}
        \item Akan digunakan metode pecahan parsial untuk menyelesaikan integral tersebut. 
        \begin{align*}
          \frac{2x^2 - 2x - 1}{x^2 + x^3} &= \frac{2x^2 - 2x - 1}{x^2 (x+ 1)} = \frac{A}{x} + \frac{B}{x^2} + \frac{C}{x+1}= \frac{Ax(x + 1) + B(x+1) + Cx^2}{x^2(x+1)} 
        \end{align*}
        Dengan menyamakan pembilang, kita dapatkan
        \[
          2x^2 - 2x - 1 = Ax(x + 1) + B(x+1) + Cx^2
        \]
        \begin{itemize}
          \item Subtitusi $x=0$
          \begin{align*}
            2(0)^2 - 2(0) - 1 &= A(0)(0 + 1) + B(0+1) + C(0)^2 \\
            -1 &= B \implies B = -1
          \end{align*}
          \item Subtitusi $x=-1$
          \begin{align*}
            2(-1)^2 - 2(-1) - 1 &= A(-1)(-1 + 1) + B(-1+1) + C(-1)^2 \\
            2 + 2 - 1 &= C \implies C = 3
          \end{align*}
          \item Subtitusi $x=1$ dan nilai variabel $B$ dan $C$ yang telah didapatkan sebelumnya
          \begin{align*}
            2(1)^2 - 2(1) - 1 &= A(1)(1 + 1) + B(1+1) + C(1)^2 \\
            2 - 2 - 1 &= A(2) + B(2) + C \\
            -1 &= 2A - 2 + 3 \implies A = -1
          \end{align*}
          Sehingga kita hanya perlu menyelesaikan integral
          \begin{align*}
            \int \left(\frac{-1}{x} + \frac{-1}{x^2} + \frac{3}{x+1}\right)dx &= \boxed{-\ln|x| + \frac{1}{x} + 3\ln|x+1| + C}
          \end{align*}
        \end{itemize}
        \item Lakukan subtitusi $u = 1 - 2\sin x$, sehingga $-\dfrac{1}{2}du = \cos x \, dx$. Untuk batas integralnya berubah sebagaimana berikut
        \begin{align*}
          x = 0 &\implies u = 1 - 2\sin(0) = 1 \\
          x = \frac{\pi}{6} &\implies u = 1 - 2\sin\left(\frac{\pi}{6}\right) = 0
        \end{align*}
        Sehingga integral tersebut dapat ditulis ulang menjadi
        \begin{align*}
          \int_0^{\frac{\pi}{6}} \frac{\cos x}{\sqrt{1 - 2 \sin x}} \, dx &= -\frac{1}{2}\int_1^0 \frac{1}{\sqrt{u}}\,du = \frac{1}{2}\int_0^1 u^{-1/2}\,du = \frac{1}{2}\left[2u^{1/2}\right]_0^1 = \boxed{1}
        \end{align*}
    \end{enumerate}

    \newpage
    \pagestyle{problems}
    
    \begin{center}
	{\underline{\textbf{\MakeUppercase{Evaluasi Tengah Semester Bersama Genap 2024/2024}}}}
    \end{center}

    \begin{center}
	\begin{tabular}{lcl}
		Mata kuliah/SKS & : & Kalkulus 2 ( SM234201 ) / 3 SKS\\
		Hari, Tanggal & : & Rabu, 24 April 2024\\
		Waktu & : & 11.00-12.40 WIB (100 menit)\\
		Sifat & : & Tertutup\\
		Kelas & : & 31-38, 104
	\end{tabular}
    \end{center}
	
    \noindent\rule{\textwidth}{2.pt}
	
    \setlength{\parindent}{5pt}
    \par Diberikan 5 soal, dengan bobot nilai masing-masing soal sama dan boleh dikerjakan tidak berurutan.
    \setlength{\parindent}{5pt}
    \centering{Tuliskan: Nama, NRP, dan Nomor Kelas pada lembar jawaban Anda.}
    \setlength{\parindent}{5pt}
    {\small
    \par \textbf{\MakeUppercase{Dilarang membawa/menggunakan kalkulator dan alat komunikasi}}
    \centering{\textbf{\MakeUppercase{dilarang memberikan/menerima jawaban selama ujian}}}}
    \par \centering{\textbf{"Setiap tindak kecurangan akan mendapat sanksi akademik."}}
	
    \noindent\rule{\textwidth}{2.pt}
	
% SOAL DI SINI YAA
\begin{enumerate}
  \item Dapatkan \( \dfrac{dy}{dx} \) dari 
  \[
  y = \frac{\left(\sqrt[3]{x^2 - 8}\right)\left(\sqrt{x^3 + 1}\right)}{x^6 - 7x + 5}
  \]
  menggunakan diferensiasi logaritmik.
  
  \item Hitung integral
  \[
  \int x \, 2^{x^2} \, dx.
  \]
  
  \item Hitung integral
  \[
  \int \sin(\ln x) \, dx.
  \]
  
  \item Hitung integral
  \[
  \int \frac{x - 4}{x^3 - x^2 + 2x} \, dx.
  \]
  
  \item Selesaikan integral tak wajar
  \[
  \int_0^2 \frac{e^x}{1 - e^x} \, dx.
  \]
\end{enumerate}
	
    \fancyfoot{
    \hrulefill\textbf{ Selamat Mengerjakan }\hrulefill

    \begin{center}
	    ``\textit{Jujur adalah kunci kesuksesan}''
    \end{center}}

    \newpage
    \pagestyle{solution}
    {\centering\textbf{SOLUSI}}

    \begin{enumerate}
        \item Logaritma-kan kedua ruas 
        \begin{align*}
          \ln y &= \ln\left(\frac{\left(\sqrt[3]{x^2 - 8}\right)\left(\sqrt{x^3 + 1}\right)}{x^6 - 7x + 5}\right) \\
          \ln y &= \ln\left(\sqrt[3]{x^2 - 8}\right) + \ln\left(\sqrt{x^3 + 1}\right) - \ln(x^6 - 7x + 5) \\
          \ln y &= \frac{1}{3}\ln(x^2 - 8) + \frac{1}{2}\ln(x^3 + 1) - \ln(x^6 - 7x + 5)
        \end{align*}
        Kemudian gunakan turunan implisit pada kedua ruas terhadap $x$.
        \begin{align*}
          \frac{1}{y}\frac{dy}{dx} &= \frac{1}{3}\cdot\frac{1}{x^2 - 8}\cdot\frac{d}{dx}(x^2 - 8) + \frac{1}{2}\cdot\frac{1}{x^3 + 1}\cdot\frac{d}{dx}(x^3 + 1) - \frac{1}{x^6 - 7x + 5}\cdot\frac{d}{dx}(x^6 - 7x + 5) \\
          \frac{1}{y}\frac{dy}{dx} &= \frac{1}{3(x^2 - 8)}(2x) + \frac{1}{2(x^3 + 1)}(3x^2) - \frac{1}{(x^6 - 7x + 5)}(6x^5 - 7)\\
          \frac{1}{y}\frac{dy}{dx} &= \frac{2x}{3(x^2 - 8)} + \frac{3x^2}{2(x^3 + 1)} - \frac{6x^5 - 7}{(x^6 - 7x + 5)}\\
          \frac{dy}{dx} &= \left[\frac{2x}{3x^2 - 24} + \frac{3x^2}{2x^3 + 2} - \frac{6x^5 - 7}{x^6 - 7x + 5}\right]\cdot y 
        \end{align*}
        Subtitusikan kembali $y=\dfrac{\left(\sqrt[3]{x^2 - 8}\right)\left(\sqrt{x^3 + 1}\right)}{x^6 - 7x + 5}$ ke dalam ekspresi diatas, sehingga didapatkan
        \begin{align*}
          \frac{dy}{dx} &= \boxed{\left[\frac{2x}{3x^2 - 24} + \frac{3x^2}{2x^3 + 2} - \frac{6x^5 - 7}{x^6 - 7x + 5}\right]\cdot\frac{\left(\sqrt[3]{x^2 - 8}\right)\left(\sqrt{x^3 + 1}\right)}{x^6 - 7x + 5}}
        \end{align*}
        \item Subtitusikan $u = x^2$, sehingga $\dfrac{1}{2}du = x \, dx$.
        \begin{align*}
          \int x \, 2^{x^2} \, dx &= \frac{1}{2}\int 2^u \, du
        \end{align*}
        Ingat bentuk eksponensial $a^x = e^{x \ln a}$ yang akibatnya
        \begin{align*}
          \frac{1}{2}\int 2^u \, du &= \frac{1}{2}\int e^{u \ln 2} \, du = \frac{1}{2\ln 2}e^{u \ln 2} + C = \frac{2^{u-1}}{\ln 2} + C = \boxed{\frac{2^{x^2 - 1}}{\ln 2} + C}
        \end{align*}
        \item Untuk soal ini dapat kita tinjau subtitusi berikut ini:
        \begin{align*}
          u &= \ln x \implies x=e^u \implies dx = e^u \, du \\
        \end{align*}
        Sehingga integralnya menjadi
        \begin{align*}
          \int \sin(\ln x) \, dx &= \int \sin(u) e^u \, du
        \end{align*}
        Disini kita gunakan integrasi parsial 
        \item .
        \item 
    \end{enumerate}

    \newpage
    \pagestyle{problems}
    
    \begin{center}
	{\underline{\textbf{\MakeUppercase{Evaluasi Tengah Semester Bersama Genap 2024/2024}}}}
    \end{center}

    \begin{center}
	\begin{tabular}{lcl}
		Mata kuliah/SKS & : & Kalkulus 2 ( SM234201 ) / 3 SKS\\
		Hari, Tanggal & : & Rabu, 24 April 2024\\
		Waktu & : & 13.30-15.10 WIB (100 menit)\\
		Sifat & : & Tertutup\\
		Kelas & : & 40-63
	\end{tabular}
    \end{center}
	
    \noindent\rule{\textwidth}{2.pt}
	
    \setlength{\parindent}{5pt}
    \par Diberikan 5 soal, dengan bobot nilai masing-masing soal sama dan boleh dikerjakan tidak berurutan.
    \setlength{\parindent}{5pt}
    \centering{Tuliskan: Nama, NRP, dan Nomor Kelas pada lembar jawaban Anda.}
    \setlength{\parindent}{5pt}
    {\small
    \par \textbf{\MakeUppercase{Dilarang membawa/menggunakan kalkulator dan alat komunikasi}}
    \centering{\textbf{\MakeUppercase{dilarang memberikan/menerima jawaban selama ujian}}}}
    \par \centering{\textbf{"Setiap tindak kecurangan akan mendapat sanksi akademik."}}
	
    \noindent\rule{\textwidth}{2.pt}
	
% SOAL DI SINI YAA
\begin{enumerate}
  \item Dapatkan turunan dari
  \[
  y = x^3 \ln(2x^2 - x)
  \]
  
  \item Hitung integral
  \[
  \int e^{2x} \sqrt{1 + e^x} \, dx.
  \]
  
  \item Hitung integral
  \[
  \int \frac{x^2}{\sqrt{4 - x^2}} \, dx.
  \]
  
  \item Hitung integral
  \[
  \int \frac{3x + 1}{(x - 2)(x + 3)(3 - 2x)} \, dx.
  \]
  
  \item Selesaikan integral tak wajar
  \[
  \int_3^4 \frac{1}{(x - 3)^2} \, dx.
  \]
\end{enumerate}
	
    \fancyfoot{
    \hrulefill\textbf{ Selamat Mengerjakan }\hrulefill

    \begin{center}
	    ``\textit{Jujur adalah kunci kesuksesan}''
    \end{center}}

    \newpage
    \pagestyle{solution}
    {\centering\textbf{SOLUSI}}

    \begin{enumerate}
        \item 
    \end{enumerate}

    \newpage

    \pagestyle{problems}

    \begin{center}
	{\underline{\textbf{\MakeUppercase{Evaluasi Tengah Semester Bersama Genap 2024/2024}}}}
    \end{center}

    \begin{center}
	\begin{tabular}{lcl}
		Mata kuliah/SKS & : & Kalkulus 2 ( SM234201 ) / 3 SKS\\
		Hari, Tanggal & : & Kamis, 25 April 2024\\
		Waktu & : & 11.00-12.40 WIB (100 menit)\\
		Sifat & : & Tertutup\\
		Kelas & : & 48-60, 107
	\end{tabular}
    \end{center}
	
    \noindent\rule{\textwidth}{2.pt}
	
    \setlength{\parindent}{5pt}
    \par Diberikan 5 soal, dengan bobot nilai masing-masing soal sama dan boleh dikerjakan tidak berurutan.
    \setlength{\parindent}{5pt}
    \centering{Tuliskan: Nama, NRP, dan Nomor Kelas pada lembar jawaban Anda.}
    \setlength{\parindent}{5pt}
    {\small
    \par \textbf{\MakeUppercase{Dilarang membawa/menggunakan kalkulator dan alat komunikasi}}
    \centering{\textbf{\MakeUppercase{dilarang memberikan/menerima jawaban selama ujian}}}}
    \par \centering{\textbf{"Setiap tindak kecurangan akan mendapat sanksi akademik."}}
	
    \noindent\rule{\textwidth}{2.pt}
	
% SOAL DI SINI YAA
\begin{enumerate}
  \item Dapatkan turunan dari
  $
  f(x) = \dfrac{e^x + \ln x}{\sinh 3x}
  $

  \item Hitung integral
  \[
  \int \frac{x + e^x}{x^2 + 2e^x} \, dx
  \]

  \item Hitung integral
  \[
  \int \frac{1}{t^{1/2} - t^{1/3}} \, dt
  \]

  \item Hitung integral
  \[
  \int \frac{t^3 + 4t^2 - t + 1}{t^3 + t^2} \, dt
  \]

  \item Dapatkan
  \[
  \lim_{x \to \infty} \left[ x \left( e^{\sin(1/x)} - 1 \right) \right]
  \]
\end{enumerate}
	
    \fancyfoot{
    \hrulefill\textbf{ Selamat Mengerjakan }\hrulefill

    \begin{center}
	    ``\textit{Jujur adalah kunci kesuksesan}''
    \end{center}}

    \newpage
    \pagestyle{solution}
    {\centering\textbf{SOLUSI}}

    \begin{enumerate}
        \item 
        \item 
        \item 
        \item 
        \item
    \end{enumerate}
    
\end{document}